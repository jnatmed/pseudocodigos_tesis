\documentclass[12pt]{article}
\usepackage[utf8]{inputenc}
\usepackage{amsmath, amssymb}
\usepackage{enumitem}
\usepackage{geometry}
\geometry{margin=2.5cm}
\title{Algoritmo: $\alpha$Distance Borderline-ADASYN-SMOTE}
\date{}

\begin{document}

\maketitle

\section*{Algoritmo: $\alpha$Distance Borderline-ADASYN-SMOTE}

\textbf{Referencia:} Feng \& Li, 2021.

\begin{enumerate}[label=\textbf{\arabic*.}]
    \item \textbf{División de vecinos:}\\
    Para cada muestra minoritaria \( p_i \), obtener sus \( m \) vecinos más cercanos y dividirlos en:
    \begin{itemize}
        \item Vecinos de clase minoritaria: cantidad \( pnum \)
        \item Vecinos de clase mayoritaria: cantidad \( nnum \)
    \end{itemize}

    \item \textbf{Cálculo de pesos inversos:}\\
    Para cada vecino \( p_j \) de \( p_i \), calcular el peso basado en la distancia:
    \[
    \alpha_j = \frac{1}{\text{dist}(p_i, p_j)}
    \]
    donde \( p_j \) es uno de los \( m \) vecinos más cercanos de \( p_i \).

    \item \textbf{Suma de pesos por clase:}
    \[
    \alpha'_p = \sum \alpha_j \text{ (de vecinos minoritarios)}, \quad 
    \alpha'_n = \sum \alpha_j \text{ (de vecinos mayoritarios)}
    \]

    \item \textbf{Identificación de muestras peligrosas:}\\
    Si \( \alpha'_n > \alpha'_p \), entonces \( p_i \) se considera una muestra \emph{peligrosa}.

    \item \textbf{Cálculo total de ejemplos sintéticos:}
    \[
    G = (N - n) \cdot \beta
    \]
    Donde:
    \begin{itemize}
        \item \( N \): número de muestras de la clase mayoritaria
        \item \( n \): número de muestras de la clase minoritaria
        \item \( \beta \in [0,1] \): proporción deseada de balance. Ejemplos:
        \begin{itemize}
            \item \( \beta = 1 \): balance total (igualar ambas clases)
            \item \( \beta = 0.4 \): generación del 40\% de la diferencia
        \end{itemize}
    \end{itemize}

    \item \textbf{Distribución proporcional del total:}\\
    Para cada muestra peligrosa \( p_i \), calcular:
    \[
    r_i = \frac{\Delta_i}{m}, \quad 
    \hat{r}_i = \frac{r_i}{\sum r_i}, \quad 
    g_i = \hat{r}_i \cdot G
    \]
    Donde:
    \begin{itemize}
        \item \( \Delta_i \): número de vecinos mayoritarios de \( p_i \)
        \item \( m \): cantidad total de vecinos
        \item \( r_i \): proporción de vecinos mayoritarios para \( p_i \)
        \item \( \hat{r}_i \): proporción normalizada
        \item \( g_i \): cantidad de muestras sintéticas a generar para \( p_i \)
    \end{itemize}

    \item \textbf{Generación de muestras sintéticas:}\\
    Para cada \( p_i \), generar \( g_i \) muestras sintéticas mediante interpolación:
    \[
    s = p_i + \lambda \cdot (p_z - p_i), \quad \lambda \in [0, 1]
    \]
    Donde \( p_z \) es un vecino minoritario aleatorio de \( p_i \).
\end{enumerate}

\end{document}
