\documentclass[12pt]{article}
\usepackage[utf8]{inputenc}
\usepackage{amsmath, amssymb}
\usepackage{enumitem}
\usepackage{geometry}
\geometry{margin=2.5cm}
\title{Algoritmo Propuesto: Sobremuestreo Híbrido $\alpha$DBASMOTE + AR-ADASYN}
\date{}

\begin{document}

\maketitle

\section*{Algoritmo: Sobremuestreo Híbrido $\alpha$DBASMOTE + AR-ADASYN}

\textbf{Fusión:} $\alpha$DBASMOTE (Feng \& Li, 2021) + AR-ADASYN (Park \& Kim, 2024)

\begin{enumerate}[label=\textbf{\arabic*.}]
    \item \textbf{Identificación de muestras peligrosas:}\\
    Para cada muestra minoritaria \( p_i \), obtener sus \( m \) vecinos más cercanos. Calcular pesos inversos:
    \[
    \alpha_j = \frac{1}{\text{dist}(p_i, p_j)}
    \]
    Dividir vecinos en mayoritarios/minoritarios y calcular:
    \[
    \alpha'_p = \sum \alpha_j \text{ (min)}, \quad \alpha'_n = \sum \alpha_j \text{ (may)}
    \]
    Si \( \alpha'_n > \alpha'_p \), se considera muestra \emph{peligrosa}.

    \item \textbf{Filtrado de ruido (mejora propia):}\\
    Se observa que algunas muestras minoritarias, aunque se ubiquen cerca de vecinos mayoritarios, tienen a sus vecinos minoritarios demasiado alejados. Esta condición sugiere que dichas muestras no representan adecuadamente a la clase minoritaria en su vecindad y podrían introducir ruido si se usan para generar datos sintéticos. 
    
    Para mitigar este efecto, se propone un filtrado basado en percentiles: se calcula el percentil 25 de todos los valores \( \alpha'_n \), que representan la suma de pesos inversos de los vecinos mayoritarios. Aquel \( p_i \) cuya \( \alpha'_n \) se encuentre por debajo de este umbral será considerada una muestra \emph{ruidosa} y se descartará del proceso de generación. Esto permite excluir las instancias menos confiables de forma estadísticamente fundamentada.
    
    \item \textbf{Cálculo global de sintéticos:}
    \[
    G = (N - n) \cdot \beta
    \]
    donde \( N \): mayoritarios, \( n \): minoritarios, \( \beta \in [0, 1] \): proporción deseada.

    \item \textbf{Distribución proporcional del total:}\\
    Para cada muestra peligrosa \( p_i \), calcular:
    \[
    r_i = \frac{\Delta_i}{m}, \quad \hat{r}_i = \frac{r_i}{\sum r_i}, \quad g_i = \hat{r}_i \cdot G
    \]
    donde \( \Delta_i \) es el número de vecinos mayoritarios.

    \item \textbf{Definición del área segura (AR-ADASYN):}\\
    Para cada \( p_i \), seleccionar dos vecinos minoritarios \( x_{nn1}, x_{nn2} \), calcular:
    \[
    \theta' = \arccos \left( \frac{(x_{nn1} - p_i) \odot (x_{nn2} - p_i)}{\|x_{nn1} - p_i\| \cdot \|x_{nn2} - p_i\|} \right)
    \]
    \[
    \theta = \min(\theta', \pi - \theta')
    \]
    \[
    r = \max(\|x_{nn1} - p_i\|, \|x_{nn2} - p_i\|)
    \]

    \item \textbf{Generación de datos sintéticos (AR-ADASYN):}\\
    Para cada \( j = 1, \dots, g_i \):
    \begin{itemize}
        \item Elegir ángulo aleatorio \( \alpha \in [0, \theta] \)
        \item Construir matriz de rotación:
        \[
        R(\alpha) =
        \begin{bmatrix}
        \cos(\alpha) & -\sin(\alpha) \\
        \sin(\alpha) & \cos(\alpha)
        \end{bmatrix}
        \]
        \item Calcular vector base \( v = x_{nn1} - p_i \)
        \item Elegir radio aleatorio \( \beta \in [0, r] \)
        \item Generar:
        \[
        x_{\text{syn}}^{(j)} = p_i + \beta \cdot R(\alpha) \cdot \frac{v}{\|v\|}
        \]
    \end{itemize}
\end{enumerate}

\end{document}
